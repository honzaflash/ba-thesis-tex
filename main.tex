%%%%%%%%%%%%%%%%%%%%%%%%%%%%%%%%%%%%%%%%%%%%%%%%%%%%%%%%%%%%%%%%%%%%
%% I, the copyright holder of this work, release this work into the
%% public domain. This applies worldwide. In some countries this may
%% not be legally possible; if so: I grant anyone the right to use
%% this work for any purpose, without any conditions, unless such
%% conditions are required by law.
%%%%%%%%%%%%%%%%%%%%%%%%%%%%%%%%%%%%%%%%%%%%%%%%%%%%%%%%%%%%%%%%%%%%

\documentclass[
  digital, %% The `digital` option enables the default options for the
           %% digital version of a document. Replace with `printed`
           %% to enable the default options for the printed version
           %% of a document.
%%  color,   %% Uncomment these lines (by removing the %% at the
%%           %% beginning) to use color in the digital version of your
%%           %% document
  table,   %% The `table` option causes the coloring of tables.
           %% Replace with `notable` to restore plain LaTeX tables.
  twoside, %% The `twoside` option enables double-sided typesetting.
           %% Use at least 120 g/m² paper to prevent show-through.
           %% Replace with `oneside` to use one-sided typesetting;
           %% use only if you don’t have access to a double-sided
           %% printer, or if one-sided typesetting is a formal
           %% requirement at your faculty.
  lof,     %% The `lof` option prints the List of Figures. Replace
           %% with `nolof` to hide the List of Figures.
  lot,     %% The `lot` option prints the List of Tables. Replace
           %% with `nolot` to hide the List of Tables.
  %% More options are listed in the user guide at
  %% <http://mirrors.ctan.org/macros/latex/contrib/fithesis/guide/mu/fi.pdf>.
]{fithesis3}
%% The following section sets up the locales used in the thesis.
\usepackage[resetfonts]{cmap} %% We need to load the T2A font encoding
\usepackage[T1,T2A]{fontenc}  %% to use the Cyrillic fonts with Russian texts.
\usepackage[
  main=english, %% By using `czech` or `slovak` as the main locale
                %% instead of `english`, you can typeset the thesis
                %% in either Czech or Slovak, respectively.
  czech         %% The additional keys allow
]{babel}        %% foreign texts to be typeset as follows:
%%
%%   \begin{otherlanguage}{czech}   ... \end{otherlanguage}
%%
%% The following section sets up the metadata of the thesis.
\thesissetup{
    date        = \the\year/\the\month/\the\day,
    university  = mu,
    faculty     = fi,
    type        = bc,
    author      = Jan Rychlý,
    gender      = m,
    advisor     = {doc. Mgr. Jan Obdržálek, PhD.},
    title       = {Game development in Haskell},
    % TeXtitle    = {Game development in Haskell},
    keywords    = {Haskell, functional paradigm, game development, Apecs},
    % TeXkeywords = {keyword1, keyword2, \ldots},
    abstract    = {%
      This is the abstract of my thesis, which can

      span multiple paragraphs.
    },
    thanks      = {%
      These are the acknowledgements for my thesis, which can

      span multiple paragraphs.
    },
    bib         = bibliography.bib,
    %% Uncomment the following line (by removing the %% at the
    %% beginning) and replace `assignment.pdf` with the filename
    %% of your scanned thesis assignment.
%%    assignment         = assignment.pdf,
}
\usepackage{makeidx}      %% The `makeidx` package contains
\makeindex                %% helper commands for index typesetting.
%% These additional packages are used within the document:
\usepackage{paralist} %% Compact list environments
\usepackage{amsmath}  %% Mathematics
\usepackage{amsthm}
\usepackage{amsfonts}
\usepackage{url}      %% Hyperlinks
\usepackage{markdown} %% Lightweight markup
\usepackage{tabularx} %% Tables
\usepackage{tabu}
\usepackage{booktabs}
\usepackage{listings} %% Source code highlighting
\lstset{
  basicstyle      = \ttfamily,
  identifierstyle = \color{black},
  keywordstyle    = \color{blue},
  keywordstyle    = {[2]\color{cyan}},
  keywordstyle    = {[3]\color{olive}},
  stringstyle     = \color{teal},
  commentstyle    = \itshape\color{magenta},
  breaklines      = true,
}
\usepackage{floatrow} %% Putting captions above tables
\floatsetup[table]{capposition=top}

% C++ macro
\newcommand{\cpp}{C\nolinebreak\texttt{+}\nolinebreak\texttt{+}}



\begin{document}
% 4/22/2021 - the introduction chapter as homework for the writing seminar
\chapter*{Introduction}
\addcontentsline{toc}{chapter}{Introduction}

Video games are a special kind of application that many consider an art form
and rewarding to develop. However, they generally involve a complex system
with a non-trivial state, a certain amount of pseudo-randomness,
and user/player input handling. This makes for non-deterministic
programs that are usually incredibly difficult to test efficiently.

Conversely, functional programming strives to eliminate
mutable state and make code more deterministic, which allows for
programs to be safer and easier to test.
These and other benefits have naturally led to people
trying out game development in functional languages, but
it remains mostly a matter of passion projects.
That said, even though the vast majority of the video game industry
still uses imperative languages like \cpp{}, the communities
\emph{are} very active, and there are hundreds of games,
blog posts, and libraries that help with
game programming in functional languages.

The focus of this thesis narrows down to exploring game development
in Haskell in the context of small-scale 2D games. The goal is
to give an overview of the process, then compare this approach
to a more conventional and imperative one
and ultimately highlight the features of Haskell that are beneficial
and those that become hurdles in the context of programming a video game.

This is done through reimplementing a single game with an already existing
imperative implementation in Haskell,
first using the Apecs\footfullcite{apecsrepo} library
and for a second time without it. After a further discussion
about chosen technologies in the following chapter,
said three implementations are described and analyzed
in chapters 2, 3, and 4. Then they are more closely compared
and the pros and cons of Haskell in game development are
evaluated and demonstrated in chapter 5.

We find that the Apecs library makes developing games
in Haskell much more approachable. On the other hand
it goes against the functional philosophy, and using it
will generally result in very imperative code wrapped in monads
that lacks the expressiveness and apparent safeness of regular Haskell.
Yet, from the second reimplementation, we learn that
some use of monads is beneficial, and it makes the code cleaner
and more elegant. In both cases, the development was
mostly a smooth experience without a single major hick-up,
unlike what often happens when dealing with a \cpp{} compiler.
Furthermore, from our small scale example,
it appears that a more "pure" architecture allows for parallel
computation resulting in better performance
than an Entity Component System like Apecs or the one
inside of Unity\footnotemark.

\footnotetext{
Disclaimer: this is only my incomplete guess of the results,
the thesis is not yet finished.
}


\chapter{The conflict of functional paradigm and video games}


\chapter{Implementing Atari Asteroids using the Apecs library}


\chapter{Implementing Atari Asteroids without using the Apecs library}


\chapter{Analyzing Atari Asteroids implementation written in C\#}


\chapter{Comparing the implementations}



\chapter*{Conclusion}
\addcontentsline{toc}{chapter}{Conclusion}




\printbibliography[heading=bibintoc] %% Print the bibliography.


\makeatletter\thesis@blocks@clear\makeatother
\phantomsection %% Print the index and insert it into the
\addcontentsline{toc}{chapter}{\indexname}
\printindex


\appendix %% Start the appendices.
\chapter{An appendix}
Here you can insert the appendices of your thesis.

\end{document}
